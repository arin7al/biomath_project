\documentclass[oneside, final, 12pt]{article}

\pagestyle{plain}

\usepackage{a4wide}
\usepackage[utf8]{inputenc}
\usepackage[russian]{babel}
\usepackage{vmargin}
\setpapersize{A4}
\setmarginsrb{2cm}{1.5cm}{1cm}{1.5cm}{5pt}{5mm}{5pt}{13mm}
\usepackage{indentfirst}
\usepackage{graphicx}

\usepackage{amsmath}
\usepackage{amsfonts}
\usepackage{amsthm}
\usepackage{amssymb}

\newcommand{\norm}[1]{\left\lVert #1 \right\rVert}

\newtheorem{definition}{Определение}
\newtheorem{feature}{Свойство}
\newtheorem{statement}{Утверждение}
\newtheorem{sthm}{Теорема}


\begin{document}


\thispagestyle{empty}

\begin{center}
\ \vspace{-3cm}

\includegraphics[width=0.5\textwidth]{msu.eps}\\
{\scshape Московский государственный университет имени М.~В.~Ломоносова}\\
Факультет вычислительной математики и кибернетики\\
Кафедра системного анализа

\vfill

\begin{LARGE}
	Отчёт по практикуму

\end{LARGE}

\vspace{1cm}

\begin{Huge}
\bfseries <<Прикладные задачи системного анализа: задачи биоматематики>>

\end{Huge}

\end{center}

\vspace{1cm}

\begin{flushright}
  \large
  \textit{Студентка 515 группы}\\
  А.\,А.~Наумова

  \vspace{5mm}

  \textit{Руководитель практикума}\\
   аспирант Д.\,А.~Алимов

\end{flushright}

\vfill

\begin{center}
Москва, 2020
\end{center}

\newpage
\tableofcontents								%	СОДЕРЖАНИЕ

\newpage
\section{Постановка задачи}						%	ПОСТАНОВКА ЗАДАЧИ
	\[
	\begin{cases}
	\dot{u} = au - \dfrac{bu^2v}{1 + Pu} + d_1u_{xx}, \\
	\dot{v} = -cv + \dfrac{du^2v}{1 + Pu} + d_2v_{xx}.
	\end{cases}
	\]

\newpage
\section{Исследование фазового портрета нераспределенной системы}						%

Сделаем замену переменных.
Пусть
\[
    \widetilde{u} = \alpha u;\\
    \widetilde{v} = \beta u;\\
    \widetilde{t} = \gamma t;\\
    \widetilde{x} = \delta x.
\]
Тогда система примет следующий вид:
\[
    \begin{cases}
        \dfrac{\gamma}{\alpha} \widetilde{u}_{\widetilde{t}} =
        \dfrac{a}{\alpha}\widetilde{u}
        - \dfrac{b}{\alpha^2\beta} \dfrac{\widetilde{u}^2\widetilde{v}}{(1 + P\widetilde{u}/\alpha)}
        + d_1\dfrac{\delta^2}{\alpha} \widetilde{u}_{\widetilde{x}\widetilde{x}}, \\

        \dfrac{\gamma}{\beta} \widetilde{v}_{\widetilde{t}} =
        \dfrac{-c}{\beta}\widetilde{v}
        + \dfrac{d}{\alpha^2\beta} \dfrac{\widetilde{u}^2\widetilde{v}}{(1 + P\widetilde{u}/\alpha)}
        + d_2\dfrac{\delta^2}{\beta} \widetilde{v}_{\widetilde{x}\widetilde{x}}.
    \end{cases}
\]

\newpage
\clearpage
\begin{thebibliography}{0}
\addcontentsline{toc}{section}{Список литературы}
	\bibitem{OC_lect}Братусь~А.\,С., Новожилов~А.\,С., Платонов~А.\,П. \label{Bratus_book}
	\emph{Динамические системы и модели биологии},
	2011 г.
\end{thebibliography}
\end{document}